$<$$<$$<$$<$$<$$<$$<$ H\+E\+A\+D \section*{\#\#\+Trabalho da Terceira Unidade E\+D\+B1/\+L\+P1\#\#}

Autores\+: Ana Clara (github.\+com/claranobre) e Rai Vitor (github.\+com/rai-\/vitor)

\section*{$\ast$$\ast$ P\+R\+O\+J\+E\+T\+O G\+E\+R\+E\+N\+C\+I\+A\+D\+O\+R D\+E D\+I\+S\+C\+O $\ast$$\ast$}

\subparagraph*{O projeto \char`\"{}\+Gerenciador de Disco\char`\"{} foi criado para pôr em prático os conceitos de Estrutura de Dados visto em sala de aula, manipulação de arquivo e operações com lista, nele é possível um usuário criar um arquivo e executar operações no disco criado}

\section*{$\ast$$\ast$\+Execução$\ast$$\ast$}

\subparagraph*{O usuário deve entrar no diretório do projeto \char`\"{}project\+\_\+disk\char`\"{} pelo terminal Linux ou Terminal simulador(\+C\+Y\+G\+W\+I\+N) do Windows}

\subparagraph*{Estando dentro do diretório o usuário deve entrar no diretório criado pela biblioteca gráfica utilizada para execução do projeto (Qt creator)\+: build-\/\+Gerenciador\+De\+Disco-\/\+Desktop\+\_\+\+Qt\+\_\+5\+\_\+4\+\_\+1\+\_\+\+G\+C\+C\+\_\+64bit-\/\+Debug }

\subparagraph*{Estando dentro do diretório o usuário deve escrever o comando \char`\"{}make\char`\"{}}

\subparagraph*{Ao terminar o comando será criado o objeto executável denominado \char`\"{}\+Gerenciador\+De\+Disco\char`\"{}}

\subparagraph*{Uma vez dentro do diretório o usuário só precisa clicar no executável ./\+Gerenciador\+De\+Disco}

\subparagraph*{Ao terminar de manipular o Gerenciador o usuário só precisa fechá-\/lo no botão de sair do software e todos os dados armazenados serão perdidos}

\section*{$\ast$$\ast$\+Verificação de Vazamento de Memória$\ast$$\ast$ }

\subparagraph*{Para verificarmos se o nosso algoritmo está com vazamento de memória de dados, utilizamos a ferramenta Valgrind para teste}

\subparagraph*{O usuário após compilar e criar o objeto deve escrever no Terminal\+:}

{\itshape valgrind --leak-\/check=full ./\+Gerenciador\+De\+Disco}

\section*{$\ast$$\ast$\+Documentação$\ast$$\ast$}

\section*{Foi utilizada a ferramenta Doxygen para auxiliar na documentação da execução desse projeto, para o usuário visualizar é necessário entrar no diretório \char`\"{}html\char`\"{} e acessar o arquivo index.\+html, logo a documentação será aberta em modo offline em seu navegador padrão}

\section*{Simulador de \hyperlink{classDisco}{Disco}}

\subsubsection*{Nome e Matrícula}

Ana Clara Nobre Mendes -\/ 2013002964 Raí Vitor Morais da Silva -\/ 2014000900

\subsubsection*{Como Executar o programa}

cd Gerenciador\+De\+Disco

./\+Gerenciador\+De\+Disco

\subsubsection*{Makefile}

Gerenciador\+De\+Disco/\+Makefile

\begin{quote}
\begin{quote}
\begin{quote}
\begin{quote}
\begin{quote}
\begin{quote}
\begin{quote}
68db918b62a0482479cdb64b5e981ab6a9dba924\end{quote}
\end{quote}
\end{quote}
\end{quote}
\end{quote}
\end{quote}
\end{quote}
